\phantomsection
\subsection*{C1A. Common Cider}
\addcontentsline{toc}{subsection}{C1A. Common Cider}

\textit{Uma \textbf{Common Cider} é produzida principalmente com maçãs de mesa (culinárias). Comparadas com a maioria dos outros estilos nesta categoria, essas sidras geralmente possuem menos taninos e mais acidez.}

\textbf{Impressão Geral}: Uma bebida refrescante, com aroma frutado e floral de maçãs, e uma acidez vibrante. Fresca, com fermentação limpa, mas que pode apresentar um leve caráter de levedura.

\textbf{Aroma e Sabor}: Caráter de maçã perceptível, seja como sabor da fruta ou como um aroma frutado-floral. Sidras doces ou de baixo teor alcoólico podem ter aroma e sabor de maçã perceptíveis. Sidras secas terão sabor mais neutro e semelhante a vinho, com alguns ésteres e notas florais derivados da maçã. Ésteres derivados da maçã não são necessariamente semelhantes à maçã; outras notas frutadas são possíveis (similar ao que ocorre quando uvas são fermentadas em vinho). Dulçor e acidez devem combinar para resultar em um caráter refrescante. Acidez de média a alta aumenta a refrescância, mas não deve ser áspera ou agressiva. Taninos moderados podem contribuir para uma maior percepção de secura no final. Fermentação geralmente limpa, sem as notas rústicas ou de MLF de algumas outras sidras regionais. Leve caráter de levedura é aceitável.

\textbf{Aparência}: De ligeiramente turva a brilhante. A cor pode variar de palha muito claro a dourado médio. Variedades de maçãs de polpa vermelha podem produzir sidras com tom rosado.

\textbf{Sensação na Boca}: Corpo de médio-baixo a médio. Leves taninos podem fornecer adstringência de baixa a média-baixa intensidade, mas com pouco amargor. Qualquer nível de carbonatação é aceito.

\textbf{Comentários}: Uma bebida refrescante, com algum corpo – nem sem graça, nem aguada. Sidras doces não devem ser enjoativas. Sidras secas não devem ser excessivamente austeras (como com sabor de fruta sutil, discreto, contido e com alta acidez). Às vezes chamada de \textit{New World Cider} ou \textit{Modern Cider}. O nome \textit{“common”} implica ausência de raridade, e não “comum” no sentido de ausência de qualidade ou classe. Uma Common Cider pode usar variedades de maçã crioulas \textit{(heirloom)}, se estas não tiverem níveis apreciáveis de taninos, caráter significativamente não frutado ou intensidade incomum – sidras com essas qualidades são melhores enquadradas em outros estilos de Tradicional Cider.

\textbf{Instruções para Inscrição}: Participantes \textbf{DEVEM} especificar tanto os níveis de carbonatação quanto de dulçor. Participantes \textbf{PODEM} especificar as variedades de maçã, particularmente se estas introduzirem características incomuns.

\textbf{Varietais}: Comuns (ex.: Winesap, McIntosh, Golden Delicious, Braeburn, Jonathan), de uso múltiplo (ex.: Northern Spy, algumas Russets, Baldwin), qualquer variedade selvagem adequada.

\textbf{Estatísticas}: OG: 1.045 - 1.065 \\
\phantom{ } \hspace{16.5mm} FG: 0.995 - 1.020 \\
\phantom{ } \hspace{16.5mm} ABV: 4.5 - 8\%

\textbf{Exemplos Comerciais}: Æppeltreow Barn Swallow Cider, Bellwether Liberty Spy, Doc’s Hard Apple Cider, Seattle Cider Dry, Tandem Ciders Smackintosh, 2 Towns BrightCider, Uncle John’s Apple Hard Cider
