\phantomsection
\subsection*{C1E. Spanish Cider}
\addcontentsline{toc}{subsection}{C1E. Spanish Cider}

\textit{\textbf{Spanish Cider} é um produto regional originário do norte da Espanha, predominantemente em Astúrias, Cantábria e País Basco. Produzida a partir de maçãs ácidas e ácidas-amargas (\textit{sharp} e \textit{bittersharp}), usando uma cofermentação natural de leveduras e bactérias. Frequentemente apresenta uma nota selvagem, com acidez volátil elevada (acetato de etila ou ácido acético), que pode ser tradicionalmente liberada usando um serviço exagerado, conhecido como \textit{Escanciar}.}

\textbf{Impressões Gerais}: Seca e fresca, com uma acidez vibrante que pode conter notas acéticas e selvagens de leves a moderadas. Impressão rústica e terrosa, tradicionalmente não filtrada.

\textbf{Aroma e Sabor}: Aromática, com notas de frutas de pomar e florais. Frequentemente apresenta um leve caráter selvagem, de estábulo ou “funky”, mas que não deve ser forte ou dominante no equilíbrio. Pode ter leve caráter de couro, especiarias ou fumaça. Sabor ácido, marcante, frequentemente com toques cítricos (limão ou toranja). Leve a moderado caráter acético e taninos são aceitáveis, mas não devem ser excessivamente avinagrado. Secura no palato e final são típicos. Notas herbais e de feno são aceitáveis. Sabores excessivamente “funky”, avinagrados ou de queijo são falhas.

\textbf{Aparência}: De clara a turva, mas geralmente turva. Coloração de palha a dourado profundo. Pode formar espuma após o serviço, mas não é persistente. Produtos tradicionais não são filtrados e praticamente não têm carbonatação depois que esta é liberada durante o serviço.

\textbf{Sensação na Boca}: Corpo médio. Produtos tradicionais têm carbonatação natural da fermentação, mas que é liberada durante o serviço, resultando em uma bebida particamente sem gás, tranquila. Porém, produtos engarrafados modernos podem ser até espumantes. Pouca ou nenhuma adstringência ou amargor, exceto nas versões Bascas.

\textbf{Comentários}: Cada região espanhola produtora de sidra tem suas próprias tradições e produtos, mas todos se enquadram nesse amplo estilo. Uma sidra Basca (Basque Cider) é mais terrosa, com notas de couro e madeira, mais amarga e mais ácida em comparação à uma sidra das Astúrias (Asturias Cider), que é mais suave, floral e frutada. Tradicionalmente fermentada lentamente em tonéis de castanheira, com notas selvagens e acéticas vindas do processo natural. Normalmente consumida jovem. As sidras tradicionais são chamadas \textit{Sidra Natural} e só têm a carbonatação residual da fermentação. Sidras espumantes são um produto moderno, feito por refermentação na garrafa. Conhecida como “Sidra” em espanhol e “Sagardoa” em basco. \textbf{Sidras que estão simplesmente contaminadas ou avinagradas não devem ser inscritas neste estilo. Se a acidez volátil for notada, juízes podem tentar liberá-la, vertendo a sidra entre copos de degustação ou usando um aerador de Spanish Cider ou de vinho. Não faça serviços teatrais em competições.}

\textbf{Instruções para Inscrição}: Participantes \textbf{DEVEM} especificar nível de carbonatação. Participantes \textbf{DEVEM} especificar dulçor, restrito de seca até média. Participantes \textbf{PODEM} especificar as variedades de maçã utilizadas; se especificadas, espera-se o caráter da varietal.

\textbf{Varietais}: Regona, Raxao, Limón Montés, Verdialona, De la Riega, San Juan, Errezil, Gezamin, Moko

\textbf{Estatísticas}: OG: 1.040 - 1.055 \\
\phantom{ } \hspace{16.5mm} FG: 0.995 - 1.010 \\
\phantom{ } \hspace{16.5mm} ABV: 5 - 6.5\%

\textbf{Exemplos Comerciais}: Barrika Basque Country Cider, El Gaitero Sidra, Fanjul Sidra Natural Llagar de Fozana, Gurutzeta Sagardo Sidra Natural, Kupela Natural Basque Cider, Mayador Sidra Natural M. Busto, Trabanco Sidra Natural, Zapiain Sidra Natural
