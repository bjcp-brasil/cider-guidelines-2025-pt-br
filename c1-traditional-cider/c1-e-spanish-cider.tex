\phantomsection
\subsection*{C1E. Spanish Cider}
\addcontentsline{toc}{subsection}{C1E. Spanish Cider}

\textit{\textbf{Spanish Cider} is a regional product originating in the north of Spain, predominantly in Asturias, Cantabria, and Basque regions. Produced from sharp and bittersharp apples using a natural co-fermentation of yeast and bacteria. Often exhibits awild note, with elevated volatile acidity (ethyl acetate or aceticacid) that traditionally is liberated using an exaggerated pourknown as Escanciar.}

\textbf{Impressões Gerais}: Dry and fresh, with a bright acidity that may contain light to moderate acetic and wild notes. Rustic and earthy impression, traditionally unfiltered.

\textbf{Aroma e Sabor}: Aromatic, with pome fruit and floral notes. Often has a light wild, barnyard, or funky quality, but this should not be strong or dominating in the balance. May have a light leather, spice, or smoke quality. Tangy, sharp, tart flavor often with citrus (lemon or grapefruit) accents. Light to moderate acetic character and tannin acceptable, but should not be overtly vinegary. Dry palate and finish are typical. Herbal and hay notes are acceptable. Excessively funky, vinegary, or cheesy flavors are faults .

\textbf{Aparência}: Clear to cloudy, but most often cloudy. Straw to deep gold in color. A head may appear after the pour, but is not persistent. Traditional products are unfiltered and virtually flat after the carbonation is liberated during the pour.

\textbf{Sensação na Boca}: Medium body. Traditional products have natural carbonation from fermentation but this is liberated during the pour to result in a nearly still drinking experience. However, modern bottled products can be up to sparkling. Little to no astringency or bitterness , except in Basque versions.

\textbf{Comentários}: Each Spanish cider-producing region has its own traditions and products, but these are combined within this broad style. Basque cider is more earthy, leathery, and woody, with more bitterness and a stronger sourness compared to the milder, floral and fruity Asturias cider. Traditionally slow - fermented in chestnut vessels with wild and acetic notes coming from the natural process. Typically enjoyed young. Traditional ciders are called Sidra Natural, and only have residual carbonation from fermentation. Sparkling sidras are a modern product using secondary refermentation in the bottle. Known as Sidra in Spanish and Sagardoa in Basque. \textbf{Ciders that are simply infected or vinegary should not be entered in this style}. If volatile acidity is noted, judges may attempt to liberate it by pouring the cider between tasting glasses or by using a Spanish Cider or wine aerator. Do not attempt theatrical pours during competitions.

\textbf{Instruções para Inscrição}: Entrants \textbf{MUST} specify carbonation level. Entrants \textbf{MUST} specify sweetness, restricted to dry through medium. Entrants \textbf{MAY} specify varieties of apples used; if specified, a varietal character will be expected.

\textbf{Varietais}: Regona, Raxao, Limón Montés, Verdialona, De la Riega, San Juan, Errezil, Gezamin, Moko

\textbf{Estatísticas}: OG: 1.040 - 1.055 \\
\phantom{ } \hspace{16.5mm} FG: 0.995 - 1.010 \\
\phantom{ } \hspace{16.5mm} ABV: 5 - 6.5\%

\textbf{Exemplos Comerciais}: Barrika Basque Country Cider, El Gaitero Sidra, Fanjul Sidra Natural Llagar de Fozana, Gurutzeta Sagardo Sidra Natural, Kupela Natural Basque Cider, Mayador Sidra Natural M. Busto, Trabanco Sidra Natural, Zapiain Sidra Natural