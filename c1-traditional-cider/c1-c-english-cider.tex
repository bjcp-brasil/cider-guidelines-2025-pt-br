\phantomsection
\subsection*{C1C. English Cider}
\addcontentsline{toc}{subsection}{C1C. English Cider}

\textit{\textbf{English Cider} is a regional product originating in the West Country, a group of counties in the southwest of England. Made from bittersweet and bittersharp apples, it is higher in tannin and lower in acidity than Common Cider. It may optionally have a phenolic-smoky character from intentional MLF. Not all cider from England fits this category; some are in the Heirloom Cider style.}

\textbf{Impressões Gerais}: Full-bodied and often seeming quite dry with a long finish from high tannin content. The fruit expression may seem subtle due to a lower estery apple character than most styles, but the fruit-derived flavor profile can be complex but non-fruity. Can optionally have a phenolic, smoky, or light barnyard MLF complexity.

\textbf{Aroma e Sabor}: The intensity of apple character tends to be subtle, but not absent. Esters and tannins can suggest apples without being overtly apple-flavored. This style often uses fruit giving significantly spicy, earthy, non-fruity flavors that are much different than those from common table apples. Acidity tends to be lower (especially if MLF has been conducted), with tannin providing much of the structure. Tannins can be moderate to high, and can add flavors reminiscent of leather, wood, dried leaves, or apple skins. MLF may add a desirable phenolic or barnyard character, with spicy, smoky, phenolic, leathery, or horsey qualities. These flavor notes are positive but are not required. If present, they must not dominate; in particular, the phenolic and farmyard notes should not be heavy. A strong farmyard character without spicy, smoky, or phenolic notes suggests a Brett contamination, which is a fault. Mousiness is a serious fault.

\textbf{Aparência}: Barely cloudy to brilliant. Medium yellow to amber color.

\textbf{Sensação na Boca}: Full body. Moderate to high tannin, perceived as astringency with some bitterness. Any carbonation level, although traditional cask versions tend to be still to moderate. Should not gush or foam.

\textbf{Comentários}: Sweeter examples exist, but dry is most traditional, particularly when considering the drying contributions of tannin.

\textbf{Instruções para Inscrição}: Entrants \textbf{MUST} specify carbonation level. Entrants \textbf{MUST} specify sweetness, restricted to dry through semi-sweet. Entrants \textbf{MAY} specify varieties of apples used; if specified, a varietal character will be expected.

\textbf{Varietais}: Kingston Black, Stoke Red, Dabinett, Porter's Perfection, Nehou, Yarlington Mill, Major, various Jerseys

\textbf{Estatísticas}: OG: 1.050 - 1.075 \\
\phantom{ } \hspace{16.5mm} FG: 0.995 - 1.015 \\
\phantom{ } \hspace{16.5mm} ABV: 6 - 9\%

\textbf{Exemplos Comerciais}: Aspall Imperial Cyder, Burrow Hill Cider Bus, Farnum Hill Farmhouse, Henney's Vintage Cider, Hogan's Dry Cider (UK), Montana CiderWorks North Fork Traditional, Oliver's Traditional Dry, Sea Cider Wild English