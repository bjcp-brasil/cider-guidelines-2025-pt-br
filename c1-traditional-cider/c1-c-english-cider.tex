\phantomsection
\subsection*{C1C. English Cider}
\addcontentsline{toc}{subsection}{C1C. English Cider}

\textit{\textbf{English Cider} é um produto regional originário de West Country, um grupo de condados ao sudoeste da Inglaterra. Produzida a partir de maçãs doces-amargas e ácidas-amargas (\textit{bittersweet} e \textit{bittersharp}), tem maior teor de taninos e menor acidez do que a Common Cider. Pode, opcionalmente, apresentar um caráter fenólico-defumado intencional de MLF. Nem toda English Cider se enquadra nesta categoria; algumas pertencem ao estilo Heirloom Cider.}

\textbf{Impressões Gerais}: Corpo cheio e muitas vezes parecendo bastante seca, com final longo devido ao alto teor de taninos. A expressão da fruta pode parecer sutil, em comparação à maioria dos estilos, devido ao menor caráter de ésteres de maçã, mas o perfil de sabor derivado da fruta pode ser complexo, mesmo que não frutado. Pode, opcionalmente, apresentar complexidade de MLF, como notas fenólicas, defumadas ou de estábulo leve.

\textbf{Aroma e Sabor}: A intensidade do caráter de maçã tende a ser sutil, mas não ausente. Ésteres e taninos podem sugerir maçãs sem apresentar um sabor marcante da fruta. Este estilo frequentemente usa frutas que conferem sabores significativamente condimentados, terrosos e não frutados, bem diferentes dos das maçãs de mesa comuns. A acidez tende a ser mais baixa (especialmente se houve MLF), com os taninos aportando grande parte da estrutura. Os taninos podem ser de moderados a altos e adicionar sabores que lembram couro, madeira, folhas secas ou cascas de maçã. A MLF pode adicionar desejável caráter fenólico ou de estábulo, com notas condimentadas, defumadas, fenólicas, de couro ou remetendo a cavalo. Estas notas de sabor são positivas, mas não obrigatórias. Se presentes, não devem dominar; em particular, as notas fenólicas e de estábulo não devem ser muito intensas. Um forte caráter de estábulo sem notas condimentadas, defumadas ou fenólicas sugere contaminação por Brett, o que é um defeito. “Mousy” é uma falha séria (veja \textit{fermentação maloláctica} em Introdução aos Estilos de Sidra e Perada).

\textbf{Aparência}: De levemente turva a brilhante. Cor de amarelo médio a âmbar.

\textbf{Sensação na Boca}: Corpo cheio. Taninos de moderados a altos, percebidos como adstringência e com algum amargor. Qualquer nível de carbonatação é aceito, embora versões tradicionais em barril tendam a ser de tranquilas a moderadas. Não deve jorrar ou espumar.

\textbf{Comentários}: Exemplares mais doces existem, mas as secas são mais tradicionais, particularmente considerando a contribuição para a secura vinda dos taninos.

\textbf{Instruções para Inscrição}: Participantes \textbf{DEVEM} especificar o nível de carbonatação. Participantes \textbf{DEVEM} especificar o dulçor, restrito de seco até meio-doce. Participantes \textbf{PODEM} especificar as variedades de maçã utilizadas; se especificadas, espera-se o caráter da varietal.

\textbf{Varietais}: Kingston Black, Stoke Red, Dabinett, Porter's Perfection, Nehou, Yarlington Mill, Major, várias Jerseys

\textbf{Estatísticas}: OG: 1.050 - 1.075 \\
\phantom{ } \hspace{16.5mm} FG: 0.995 - 1.015 \\
\phantom{ } \hspace{16.5mm} ABV: 6 - 9\%

\textbf{Exemplos Comerciais}: Aspall Imperial Cyder, Burrow Hill Cider Bus, Farnum Hill Farmhouse, Henney's Vintage Cider, Hogan's Dry Cider (UK), Montana CiderWorks North Fork Traditional, Oliver's Traditional Dry, Sea Cider Wild English
