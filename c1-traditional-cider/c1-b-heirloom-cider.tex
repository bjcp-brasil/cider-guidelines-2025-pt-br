\phantomsection
\subsection*{C1B. Heirloom Cider}
\addcontentsline{toc}{subsection}{C1B. Heirloom Cider}

\textit{\textbf{Heirloom Cider} é um estilo de definição ampla, que constuma usar pelo menos algumas maçãs de sidra, a fim de criar um produto com mais taninos do que a Common Cider. Geralmente é produzido fora das regiões associadas aos estilos English, French e Spanish Cider, e não possui as características distintas de MLF ou rústicas desses estilos. É um tipo de sidra “artesanal”, produzida na América do Norte, no leste da Inglaterra e em outras partes do mundo.}

\textbf{Impressões Gerais}: Une o caráter de maçã e a acidez de uma Common Cider com a presença de taninos típica das English Ciders e French Ciders, enquanto mantém um perfil limpo de fermentação.

\textbf{Aroma e Sabor}: A intensidade do caráter de maçã, ésteres e adocicado geralmente varia de acordo com o nível de dulçor. Variedades (\textit{heirloom}) para sidra podem trazer suas próprias qualidades únicas, frequentemente rústicas. A acidez pode ser de moderada a alta. Os taninos podem variar de médio-baixo a médio-alto. Os taninos podem aumentar a impressão de secura no final, enquanto podem contribuir com sabores que lembram madeira, couro ou casca de maçã. Acidez e taninos trazem equilíbrio ao dulçor e dão estrutura à sidra; ambos estão tipicamente presentes, sem necessidade de estarem na mesma intensidade. Possui um perfil de fermentação limpa, sem fenóis derivados de MFL ou caráter como de estábulo. O \textit{off-flavor “mousy”} é uma falha séria. Leve caráter de levedura é aceitável.

\textbf{Aparência}: De ligeiramente turva a brilhante. Cor variando de palha a dourado profundo. Variedades de maçãs de polpa vermelha podem produzir sidras com tom rosado.

\textbf{Sensação na Boca}: Corpo de médio a cheio, dependendo do nível de taninos. Qualquer adstringência ou amargor proveniente dos taninos deve ser, no máximo, moderado. Qualquer nível de carbonatação é aceito.

\textbf{Comentários}: Aproxima-se mais próxima da English Cider, mas sem fenóis de MLF ou caráter de estábulo, e com equilíbrio de acidez mais elevado. Às vezes chamada de Heritage Cider ou Traditional Cider. O nome \textit{“heirloom”} implica o uso de variedades mais antigas e pouco cultivadas de maçãs para sidra, não que haja algum prestígio adicional, especialmente em relação à Common Cider.

\textbf{Instruções para Inscrição}: Os participantes \textbf{DEVEM} especificar os níveis de carbonatação e de dulçor. Os participantes \textbf{PODEM} especificar as variedades de maçã utilizadas; se especificadas, espera-se o caráter da varietal.

\textbf{Varietais}: Variedades de uso múltiplo da Common Cider e muitas das mesmas variedades doces-amargas e ácidas-amargas (\textit{bittersweet} e \textit{bittersharp}) usadas em English Ciders e French Ciders, ou outras variedades \textit{heirloom} ou próprias para sidra, maçãs silvestres, varietais híbridos, selvagens tânicas.

\textbf{Estatísticas}: OG: 1.050 - 1.080 \\
\phantom{ } \hspace{16.5mm} FG: 0.995 - 1.020 \\
\phantom{ } \hspace{16.5mm} ABV: 6 - 9\%

\textbf{Exemplos Comerciais}: Eve's Cidery Autumn's Gold, Farnum Hill Extra Dry, Redbyrd Orchard Cloudsplitter, Sea Cider Flagship, Snowdrift Cliffbreaks Blend, Tandem Ciders Crabster, West County Cider Redfield
