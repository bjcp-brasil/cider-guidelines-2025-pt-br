\phantomsection
\subsection*{C1B. Heirloom Cider}
\addcontentsline{toc}{subsection}{C1B. Heirloom Cider}

\textit{\textbf{Heirloom Cider} é um estilo de definição ampla, que constuma usar pelo menos algumas maçãs de sidra, a fim de criar um produto com mais taninos do que a Common Cider. Geralmente é produzido fora das regiões associadas aos estilos English, French e Spanish Cider, e não possui as características distintas de MLF ou rústicas desses estilos. É um tipo de sidra “artesanal”, produzida na América do Norte, no leste da Inglaterra e em outras partes do mundo.}

\textbf{Impressões Gerais}: Une o caráter de maçã e a acidez de uma Common Cider com a presença de taninos típica das English e French Ciders, enquanto mantém um perfil limpo de fermentação.

\textbf{Aroma e Sabor}: A intensidade do caráter de maçã, ésteres e adocicado geralmente varia de acordo com o nível de dulçor. Variedades tradicionais (\textit{heirloom}) para sidra podem trazer suas próprias qualidades únicas, frequentemente rústicas. A acidez pode ser de moderada a alta. Os taninos podem variar de médio-baixo a médio-alto. Os taninos podem aumentar a impressão de secura no final, enquanto podem contribuir com sabores que lembram madeira, couro ou casca de maçã. Acidez e taninos trazem equilíbrio ao dulçor e dão estrutura à sidra; ambos estão tipicamente presentes, sem necessidade de estarem na mesma intensidade. Possui um perfil de fermentação limpa, sem fenóis derivados de MFL ou caráter como de estábulo. O off-flavor “mousy” é uma falha séria. Leve caráter de levedura é aceitável.

\textbf{Aparência}: Slightly cloudy to brilliant. Color ranges from straw to deep gold. Red-fleshed apple varieties can produce ciders with a blush hue.

\textbf{Sensação na Boca}: Medium to full body, depending on tannin level. Any astringency and bitterness from tannin should be no more than moderate. Any level of carbonation.

\textbf{Comentários}: Probably most similar to English Cider, but without any MLF phenols or barnyard character, and having a higher acid balance. Sometimes called Heritage Cider or Traditional Cider. The name heirloom implies the use of older, not-widely-grown cider apple varieties, not that there is some added prestige, especially relative to Common Cider.

\textbf{Instruções para Inscrição}: Entrants \textbf{MUST} specify both carbonation and sweetness levels. Entrants \textbf{MAY} specify varieties of apples used; if specified, a varietal character will be expected.

\textbf{Varietais}: Multi-use varieties from Common Cider and many of the same bittersweet and bittersharp varieties used in English or French Ciders, or other heirloom or cider varieties, crabapples, hybrids, tannic wildings

\textbf{Estatísticas}: OG: 1.050 - 1.080 \\
\phantom{ } \hspace{16.5mm} FG: 0.995 - 1.020 \\
\phantom{ } \hspace{16.5mm} ABV: 6 - 9\%

\textbf{Exemplos Comerciais}: Eve's Cidery Autumn's Gold, Farnum Hill Extra Dry, Redbyrd Orchard Cloudsplitter, Sea Cider Flagship, Snowdrift Cliffbreaks Blend, Tandem Ciders Crabster, West County Cider Redfield
