\phantomsection
\subsection*{C1D. French Cider}
\addcontentsline{toc}{subsection}{C1D. French Cider}

\textit{\textbf{French Cider} is a regional product originating in the northwest of France, predominantly Normandy and Brittany. Made using bittersweet and bittersharp apples, it can have a higher tannin level, but it is often made sweeter to balance. The French also use MLF as do the English, but the character is often lower. Salts may be adjusted and nutrients may be deprived to arrest fermentation.}

\textbf{Impressões Gerais}: Medium to sweet, full-bodied, rich. Somewhat fruity. May have a background level of phenolic, smoky, or farmyard character.

\textbf{Aroma e Sabor}: Fruity, often fairly sweet with a full, rich flavor. The sweetness and tannin combine to give a palate fullness that often delivers considerable apple flavor. Tannins can dry the finish slightly. MLF notes of spicy-smoky, phenolic, lightly funky, and farmyard are common but not required (just as with English Cider), but must not be pronounced if present.

\textbf{Aparência}: Clear to brilliant. Medium yellow to amber color. Color may be deeper than other traditional styles. Higher carbonation levels may create a brief foam stand similar to soda.

\textbf{Sensação na Boca}: Medium to full body, mouth-filling. Moderate tannin, perceived mainly as palate fullness and astringency rather than bitterness. Carbonation moderate to champagne-like, but at higher levels it must not gush or foam.

\textbf{Comentários}: Typically made sweet to balance the tannin levels from the traditional apple varieties. The French technique of défécation (keeving in English) can be used to slow fermentation by depriving nutrients. Some may approximate this by backsweetening with juice. Commercial examples are frequently carbonated in the bottle. The French expect a subtler MLF character than do the English. Known as Cidre in French, and often sold by sweetness level.

\textbf{Instruções para Inscrição}: Entrants MUST specify carbonation level. Entrants MUST specify sweetness, restricted to medium through sweet. Entrants MAY specify varieties of apples used; if specified, a varietal character will be expected.

\textbf{Varietais}: Nehou, Muscadet de Dieppe, Reine des Pommes, Michelin

\textbf{Estatísticas}: OG: 1.045 – 1.065 \\
\phantom{ } \hspace{16.5mm} FG: 1.005 – 1.020 \\
\phantom{ } \hspace{16.5mm} ABV: 3 – 6\%

\textbf{Exemplos Comerciais}: Examples: Bellot Vintage Cider, Domaine Dupont Cidre Bouché, Écusson Cidre Bio Doux, Eric Bordelet Sidre Tendre, Etienne Dupoint Brut, Maison Hérout Cuvée Tradition