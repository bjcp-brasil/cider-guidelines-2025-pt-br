\phantomsection
\subsection*{C1D. French Cider}
\addcontentsline{toc}{subsection}{C1D. French Cider}

\textit{\textbf{French Cider} é um produto regional originário do noroeste da França, predominantemente da Normandia e da Bretanha. Produzida com maçãs doces-amargas e ácidas-amargas (\textit{bittersweet} e \textit{bittersharp}), pode apresentar teor alto de taninos, mas frequentemente é produzida mais doce para equilibrá-los. A French Cider, assim como a English, também utiliza MLF, mas o caráter é geralmente menor. Os sais podem ser ajustados e os nutrientes podem ser reduzidos para interromper a fermentação.}

\textbf{Impressões Gerais}: Dulçor de médio a doce, encorpada, rica. Um tanto quanto frutada. Pode ter de fundo um caráter fenólico, defumado ou de estábulo.

\textbf{Aroma e Sabor}: Frutada, muitas vezes bastante doce, com sabor cheio e rico. O dulçor e os taninos se combinam para preencher o palato, frequentemente trazendo um considerável sabor de maçã. Os taninos podem secar levemente o final. Notas condimentadas-defumadas, fenólicas, levemente “funky” e de estábulo do processo de MLF são comuns, mas não obrigatórias (assim como na English Cider), e, se presentes, não devem ser pronunciadas.

\textbf{Aparência}: Limpa a brilhante. Cor de amarelo médio a âmbar. A cor pode ser mais profunda que em outros estilos tradicionais. Níveis mais altos de carbonatação podem criar uma espuma breve, semelhante à de refrigerante.

\textbf{Sensação na Boca}: Corpo médio a cheio, preenchendo a boca. Taninos moderados, percebidos principalmente como preenchimento no palato e como adstringência, mais do que como amargor. Carbonatação de moderada a tipo champanhe, mas em níveis mais altos não deve jorrar ou espumar.

\textbf{Comentários}: Normalmente produzida doce para equilibrar os níveis de taninos das variedades tradicionais de maçãs. A técnica francesa de \textit{défécation} (em inglês, \textit{keeving}) pode ser usada para desacelerar a fermentação, privando-a de nutrientes. Alguns podem aproximar-se disso adoçando novamente com suco (\textir{backsweetening}). Exemplos comerciais são frequentemente carbonatados em garrafa. Espera-se um caráter mais sutil de MLF nas French Ciders que nas English Ciders. Conhecida como \textit{“Cidre”}, em francês, é frequentemente comercializada com base em seu nível de dulçor.

\textbf{Instruções para Inscrição}: Os participantes \textbf{DEVEM} especificar o nível de carbonatação. Os participantes \textbf{DEVEM} especificar o dulçor, restrito de médio até doce. Os participantes \textbf{PODEM} especificar as variedades de maçã utilizadas; se especificadas, espera-se o caráter da varietal.

\textbf{Varietais}: Nehou, Muscadet de Dieppe, Reine des Pommes, Michelin

\textbf{Estatísticas}: OG: 1.045 – 1.065 \\
\phantom{ } \hspace{16.5mm} FG: 1.005 – 1.020 \\
\phantom{ } \hspace{16.5mm} ABV: 3 – 6\%

\textbf{Exemplos Comerciais}: Examples: Bellot Vintage Cider, Domaine Dupont Cidre Bouché, Écusson Cidre Bio Doux, Eric Bordelet Sidre Tendre, Etienne Dupoint Brut, Maison Hérout Cuvée Tradition
