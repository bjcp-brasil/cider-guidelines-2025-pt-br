\phantomsection
\subsection*{C4A. Common Perry}
\addcontentsline{toc}{subsection}{C4A. Common Perry}

\textit{\textbf{Common Perry} é feita a partir de peras de mesa (culinárias).}

\textbf{Impressão Geral}: Levemente frutada, corpo mais cheio. Normalmente de meio-seca a meio-doce. Pode variar de tranquila até frisante. Apenas uma acetificação muito sutil é aceitável.

\textbf{Aroma e Sabor}: Apresenta caráter frutado de pera, que pode ser suave, mas tende a se intensificar em versões mais doces. O caráter de pera reflete os sabores esperados da fermentação de peras de mesa, que nem sempre lembram o gosto da fruta fresca. Versões mais secas tendem a ter perfil semelhante a um vinho branco jovem. O nível de acidez deve ser de suave a equilibrado, nunca agressivo. Os taninos podem variar de leves a equilibrados, mas não devem adicionar amargor significativo. O equilíbrio entre acidez e taninos é variável, mas geralmente vai de equilibrado a levemente ácido. Não deve apresentar caráter “mousy” (veja \textit{fermentação maloláctica} em Introdução aos Estilos de Sidra e Perada), viscoso ou oleoso.

\textbf{Aparência}: De levemente turva a límpida. Geralmente bastante clara, com cor que vai do palha ao dourado.

\textbf{Sensação na Boca}: Corpo relativamente cheio. Taninos de baixos a moderados, perceptíveis como adstringência. Carbonatação de tranquila até espumante, embora a maioria não ultrapasse o nível de frisante.

\textbf{Comentários}: Comparada à Heirloom Perry, a Common Perry tem menos taninos, maior caráter de fruta de mesa e pode apresentar mais acidez. Algumas peras de mesa contêm quantidades significativas de sorbitol, o que pode dar a uma perada seca uma impressão adocicada. A percepção do sorbitol como doce varia muito de pessoa para pessoa. Portanto, os participantes devem especificar o nível de dulçor de acordo com a quantidade real de açúcar residual, e os juízes precisam estar cientes de que podem perceber mais dulçor do que o declarada. O uso de suco \textit{in natura} de pera para alcançar um perfil de sabor reconhecível (\textit{back-sweetening}) pode ser encontrado em exemplos comerciais, mas não é necessariamente autêntico ou esperado em regiões com tradição longa e contínua de produção. O termo “commom” implica ausência de raridade, e não “comum” no sentido de ausência de qualidade ou classe. Às vezes também chamada de \textit{New World Perry} ou \textit{Modern Perry}.

\textbf{Instruções para Inscrição}: Participantes \textbf{DEVEM} especificar tanto o nível de carbonatação quanto o de dulçor.

\textbf{Varietais}: Bartlett, Kiefer, Comice, Conference

\textbf{Estatísticas}: OG: 1.050 - 1.060 \\
\phantom{ } \hspace{16.5mm} FG: 1.000 - 1.020 \\
\phantom{ } \hspace{16.5mm} ABV: 5 - 8\%

\textbf{Exemplos Comerciais}: Æppeltreow Perry, EdenVale Pear Cider, Seattle Cider Perry, Snowdrift Semi-Dry Perry, Twin Pines Hammer Bent Perry, Uncle John's Perry
