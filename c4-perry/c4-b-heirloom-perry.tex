\phantomsection
\subsection*{C4B. Heirloom Perry}
\addcontentsline{toc}{subsection}{C4B. Heirloom Perry}

\textit{Uma perada tradicional feita a partir de “peras de perada” cultivadas especificamente para esse fim, e não para consumo \textit{in natura} ou para culinária. Muitas dessas variedades são quase intragáveis, devido ao alto teor de taninos; algumas também são bastante duras. As peras de perada podem conter quantidades significativas de sorbitol, um álcool de açúcar não fermentável com sabor adocicado. Por isso, uma perada pode transmitir a impressão de dulçor e ainda assim ser completamente seca (sem açúcar residual).}

\textbf{Impressões Gerais}: Tânica e levemente frutada, com corpo mais cheio. Exemplos ingleses tendem a ser mais secos que os franceses, portanto o nível de dulçor pode variar. Exemplos ingleses e franceses podem apresentar maior nível de carbonatação.

\textbf{Aroma e Sabor}: Apresenta caráter de pera fermentada perceptível, que pode variar de sutil a bastante frutado. O caráter de pera costuma ser mais complexo do que em uma Common Perry e não lembra fortemente peras de mesa. A impressão geralmente remete a um vinho branco jovem. Um leve amargor tânico é possível. O nível de acidez deve ser equilibrado, não agressivo, já que normalmente há mais taninos que acidez. O sorbitol pode contribuir para a impressão de dulçor. Não deve apresentar caráter “mousy” (veja \textit{fermentação maloláctica} em Introdução aos Estilos de Sidra e Perada), viscoso ou oleoso. A perada pode, às vezes, ter um nível muito baixo de acetificação natural, o que não está relacionado a contaminação.

\textbf{Aparência}: De levemente turva a límpida. Geralmente bastante pálida, variando do palha ao dourado. Carbonatação de tranquila até espumante, embora a maioria não passe de frisante.

\textbf{Sensação na Boca}: Corpo relativamente cheio. Taninos de moderados a altos, perceptíveis como adstringência. O sorbitol pode conferir textura suave e macia. Não deve parecer como xarope.

\textbf{Comentários}: Comparada à Common Perry, a Heirloom Perry é mais marcada pelos taninos, pode apresentar algum amargor e tem sabor de pera mais complexo. É importante observar que uma perada seca pode transmitir impressão adocicada devido ao sorbitol presente nas peras, e a percepção desse dulçor varia muito de pessoa para pessoa. Portanto, os participantes devem especificar o nível de dulçor de acordo com a quantidade real de açúcar residual, e os juízes precisam estar cientes de que podem perceber mais dulçor que a declarado. O uso de suco \text{in natura} de pera para alcançar um perfil de sabor reconhecível pode ser encontrado em alguns exemplos comerciais, mas isso não é necessariamente autêntico ou esperado em regiões com tradição longa e contínua. Às vezes também chamada de \textit{Traditional Perry} ou \textit{Heritage Perry}. O nome “heirloom” (herança) implica o uso de variedades mais antigas e pouco cultivadas de peras para perada, não que haja algum prestígio adicional, especialmente em relação à Common Perry.

\textbf{Instruções para Inscrição}: Participantes \textbf{DEVEM} especificar tanto o nível de carbonatação quanto o de dulçor.

\textbf{Varietais}: Butt, Gin, Brandy, Barland, Blakeney Red, Thorn, Moorcroft

\textbf{Estatísticas}: OG: 1.050 - 1.070 \\
\phantom{ } \hspace{16.5mm} FG: 1.000 - 1.020 \\
\phantom{ } \hspace{16.5mm} ABV: 4 - 9\%

\textbf{Exemplos Comerciais}: Æppeltreow Orchard Oriole Perry, Burrow Hill Perry, Christian Drouin Poiré, Dragon's Head Sparkling Perry, Eric Bordelet Poiré Authentique, EZ Orchards Poire, Hogan's Classic Perry (UK), Oliver's Classic Perry
