\phantomsection
\subsection*{C4B. Heirloom Perry}
\addcontentsline{toc}{subsection}{C4B. Heirloom Perry}

\textit{Uma Perada tradicional feita a partir de “peras de perada” cultivadas especificamente para esse fim, e não para consumo in natura ou para culinária. Muitas dessas variedades são quase intragáveis devido ao alto teor de taninos; algumas também são bastante duras. As peras de perada podem conter quantidades significativas de sorbitol, um álcool de açúcar não fermentável com sabor adocicado. Por isso, uma perada pode transmitir a impressão de doçura e ainda assim ser completamente seca (sem açúcar residual).}

\textbf{Impressões Gerais}: Tânica e levemente frutada, com corpo mais cheio. Exemplos ingleses tendem a ser mais secos que os franceses, portanto o nível de dulçor pode variar. Exemplos ingleses e franceses podem apresentar maior nível de carbonatação.

\textbf{Aroma e Sabor}: Apresenta caráter de pera fermentada perceptível, que pode variar de sutil a bastante frutado. O caráter de pera costuma ser mais complexo do que em uma Common Perry e não lembra fortemente peras de mesa. A impressão geralmente remete a um vinho branco jovem. Um leve amargor tânico é possível. O nível de acidez deve ser equilibrado, não agressivo, já que normalmente há mais tanino que acidez. O sorbitol pode contribuir para a impressão de dulçor. Não deve apresentar caráter “mousy” (veja \textit{fermentação malolática} em Introdução aos Estilos de Sidra e Perada), viscoso ou oleoso. A perada pode, às vezes, ter um nível muito baixo de acetificação natural, o que não está relacionado a contaminação.

\textbf{Aparência}: De levemente turva a límpida. Geralmente bastante pálida, variando do palha ao dourado. Carbonatação de tranquila até espumante, embora a maioria não passe de nível médio.

\textbf{Sensação na Boca}: Corpo relativamente cheio. Taninos de moderado a alto, perceptíveis como adstringência. O sorbitol pode conferir textura suave e macia. Não deve parecer como xarope.

\textbf{Comentários}: Compared to Common Perry, Heirloom Perry is more tannin-forward, may have some bitterness, and has a more complex pear flavor. Note that a dry perry may give an impression of sweetness due to sorbitol in the pears, and perception of sorbitol as sweet is highly variable from one person to another. Hence entrants should specify sweetness according to actual residual sugar amount, and judges must be aware that they might perceive more sweetness than how the perry was entered. Back-sweetening with raw pear juice to achieve a recognizable flavor profile can be found in some commercial examples, but this is not necessarily authentic or expected in perry from areas with a long, continuous tradition. Sometimes called Traditional Perry or Heritage Perry. The name heirloom implies the use of older, not-widely-grown perry pear varieties, not that there is some added prestige, especially relative to Common Perry.

\textbf{Comentários}: Comparada à Common Perry, a Heirloom Perry é mais marcada pelos taninos, pode apresentar alguma amargura e tem sabor de pera mais complexo. É importante observar que uma perada seca pode transmitir impressão de doçura devido ao sorbitol presente nas peras, e a percepção desse dulçor varia muito de pessoa para pessoa. Portanto, os participantes devem especificar o nível de dulçor de acordo com a quantidade real de açúcar residual, e os juízes precisam estar cientes de que podem perceber mais doçura do que a declarada. Adoçar novamente com suco in natura de pera para obter um perfil de sabor reconhecível pode ser encontrado em alguns exemplos comerciais, mas isso não é necessariamente autêntico ou esperado em regiões com tradição longa e contínua. Às vezes também chamada de Traditional Perry ou Heritage Perry. O nome “heirloom” (herança) implica no uso de variedades antigas de peras de perada, não amplamente cultivadas, e não com algum prestígio adicional, especialmente em relação à Common Perry.

\textbf{Instruções para Inscrição}: Os participantes \textbf{DEVEM} especificar tanto o nível de carbonatação quanto o de dulçor.

\textbf{Varietais}: Butt, Gin, Brandy, Barland, Blakeney Red, Thorn, Moorcroft

\textbf{Estatísticas}: OG: 1.050 - 1.070 \\
\phantom{ } \hspace{16.5mm} FG: 1.000 - 1.020 \\
\phantom{ } \hspace{16.5mm} ABV: 4 - 9\%

\textbf{Exemplos Comerciais}: Æppeltreow Orchard Oriole Perry, Burrow Hill Perry, Christian Drouin Poiré, Dragon's Head Sparkling Perry, Eric Bordelet Poiré Authentique, EZ Orchards Poire, Hogan's Classic Perry (UK), Oliver's Classic Perry