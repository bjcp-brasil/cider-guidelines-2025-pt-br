\phantomsection
\subsection*{C4C. Ice Perry}
\addcontentsline{toc}{subsection}{C4C. Ice Perry}

\textit{Um estilo regional (\textit{Poiré de Glace}) originário do Quebec nos anos 2000, muitas vezes produzido por sidrarias ou locais que também elaboram sidras usando processo semelhante. O suco de pera é congelado antes da fermentação para concentrar os açúcares. A fermentação geralmente é interrompida antes da conclusão para atingir o nível desejado de dulçor. Não é permitido o uso de fontes de açúcares para ajustar a densidade inicial ou final.}

\textbf{Impressão Geral}: Corpo cheio, brilhante e frutada, com acidez equilibrada. Doce, mas não enjoativa. De tranquila a frisante.

\textbf{Aroma e Sabor}: Frutado, macio, doce-ácido. O aroma da fruta é vívido e intenso, muitas vezes lembrando peras cozidas, compotas ou pêssegos cristalizados. Os sabores frutados de leves a moderados podem remeter a peras cozidas, secas, em conserva, cristalizadas ou caramelizadas. O final macio e arredondado pode apresentar notas adicionais de mel, nozes, confeitaria ou frutas tropicais. A acidez deve ser suficiente para evitar uma sensação enjoativa. Pode ter taninos (adstringentes ou amargos), mas apenas de leves a moderados. Não deve apresentar caráter “mousy” (veja \textit{fermentação maloláctica} em Introdução aos Estilos de Sidra e Perada), viscoso ou oleoso. Presença perceptível de acetona é uma falha. Apenas uma acetificação muito sutil é aceitável.

\textbf{Aparência}: De límpida a cristalina. Cor de dourado a âmbar. Normalmente tranquila, mas é permitido leve indício de carbonatação.

\textbf{Sensação na Boca}: Corpo cheio. Textura macia e suave, com final muito longo e sedoso. O aquecimento alcoólico vai de leve a moderadamente baixo, podendo não ser perceptível devido ao dulçor. A maioria dos exemplos é tranquila, sem gás, mas uma leve carbonatação é aceitável.

\textbf{Comentários}: Embora originária do Canadá, não é um produto exclusivamente canadense. O nível de taninos e acidez é visivelmente menor do que em uma Ice Cider.

\textbf{Instruções para Inscrição}: Participantes \textbf{DEVEM} especificar a densidade inicial, a densidade final ou quantidade de açúcar residual, o teor alcoólico e o nível de carbonatação.

\textbf{Varietais}: Bartlett, Bosc, Flemish Beauty, other table pears

\textbf{Estatísticas}: OG: 1.130 - 1.170 \\
\phantom{ } \hspace{16.5mm} FG: 1.050 - 1.085 \\
\phantom{ } \hspace{16.5mm} ABV: 9 - 12\%

\textbf{Exemplos Comerciais}: Coteau Rougemont Poiré de Glace, Domaine de la Galotière Poiré de Glace, Domaine de Lavoie Poiré de Glace, Vergers Écologiques Philion Gaia, Domaine des Salamandres Le Classique
