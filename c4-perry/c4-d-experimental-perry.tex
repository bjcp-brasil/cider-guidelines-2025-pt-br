\phantomsection
\subsection*{C4D. Experimental Perry}
\addcontentsline{toc}{subsection}{C4D. Experimental Perry}

\textit{Esta é uma categoria aberta, abrangente, para peradas com outros ingredientes ou produzidas com processos diferentes que resultem em um produto que não se encaixe em nenhum outro estilo da categoria C4, como versões à base de pera das categorias C3A e C3B (peradas com frutas ou condimentos). Também pode ser usada para qualquer outro tipo de perada histórica ou regional tradicional ainda não descrita, ou para peradas que atendam às definições existentes das diretrizes, mas que estejam visivelmente fora dos parâmetros listados (ex.: teor alcoólico, dulçor, carbonatação). Se a perada se encaixar em um estilo já definido, então não é considerada uma Experimental Perry.}
\textit{Produtos derivados de outras frutas de pomar (ex.: marmelo), incluindo aquelas de frutas semelhantes a bagas do gênero \textit{Amelanchier genus} (ex.: juneberry (ameixa-de-junho), serviceberry (amelanche) e saskatoon berry (saskatoon)), podem ser incluídos aqui no lugar de uma categoria separada, desde que a fruta experimental seja dominante na formulação.}

\textbf{Aroma e Sabor}: O caráter de perada deve estar sempre presente e deve estar em harmonia com os ingredientes adicionados ou os efeitos do processo. Se um barril previamente usado para destilados for utilizado, o caráter do destilado (rum, uísque, etc.) pode variar de sutil (quase imperceptível) até equilibrado e complementar (sem dominar ou sobrepujar o caráter da perada). O equilíbrio geral e a boa drinkability são fatores críticos de sucesso neste estilo. A perada resultante deve conter elementos experimentais reconhecíveis e ser agradável de beber.

\textbf{Aparência}: De límpida a brilhante. A cor deve ser a de uma perada padrão, a menos que os ingredientes ou processos declarados contribuam para a cor.

\textbf{Sensação na Boca}: Se um estilo-base foi declarado, o corpo e a sensação devem refletir esse estilo. Ingredientes ou processos declarados podem resultar em corpo adicional, ou em maior presença de taninos, adstringência, amargor ou outras características.

\textbf{Comentários}: Se for usada uma mistura de frutas, o caráter de perada deve permanecer dominante. Independentemente da natureza experimental, a bebida resultante deve ser reconhecível como uma perada. A descrição da perada é uma informação crítica para os juízes e deve ser suficiente para que compreendam o conceito. Se ingredientes especiais forem declarados, eles devem ser perceptíveis (exceção: potenciais alergênicos não precisam ser perceptíveis, mas devem ser declarados).

\textbf{Instruções para Inscrição}: Os participantes DEVEM especificar os ingredientes ou processos que tornam a amostra uma perada experimental. Os participantes DEVEM especificar os níveis de carbonatação e dulçor. Os participantes PODEM especificar um estilo-base ou fornecer uma descrição mais detalhada do conceito.

\textbf{Varietais}: Qualquer.

\textbf{Estatísticas}: OG: 1.045 - 1.100 \\
\phantom{ } \hspace{16.5mm} FG: 0.995 - 1.020 \\
\phantom{ } \hspace{16.5mm} ABV: 5 - 12\%

\textbf{Exemplos Comerciais}: Æppeltreow Pear Wine, Sea Cider Ginger Perry, Snow Capped Cider JalaPEARño