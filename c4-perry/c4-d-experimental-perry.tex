\phantomsection
\subsection*{C4D. Experimental Perry}
\addcontentsline{toc}{subsection}{C4D. Experimental Perry}

\textit{This is an open-ended, catch-all category for perry with other ingredients or for perry using other processes that result in a product not fitting any other C4 styles, such as pear-based versions of C3A and C3B (fruited or spiced perry). It may also be used for any other type of historical or regional traditional perry not already described, or for perry that otherwise meets existing guideline definitions, except that it is noticeably outside listed style parameters (e.g., strength, sweetness, carbonation). If the perry fits a previously defined style, then it is not an Experimental Perry.}
\textit{Products derived from other pome fruit (e.g., quince) including those berry-like fruit in the Amelanchier genus (e.g., juneberry, serviceberry, saskatoon berry) may be entered here in lieu of a separate category, provided the experimental fruit is dominant in the formulation.}

\textbf{Aroma e Sabor}: The perry character must always be present, and must fit with added ingredients or process effects. If a spirit barrel was used, the character of the spirit (rum, whiskey, etc.) may range from subtle (barely recognizable) to balanced and complementary (short of dominating and overwhelming the perry character). Overall balance and drinkability are the critical success factors for this style. The resulting perry should contain recognizable experimental components, and be pleasant to drink.

\textbf{Aparência}: Clear to brilliant. Color should be that of a standard perry unless other declared ingredients or processes contribute color.

\textbf{Sensação na Boca}: If a base style has been declared, the body and mouthfeel should be reflective of that style. Declared ingredients or processes may result in additional body, or in increased tannic, astringent, bitter, or other characteristics.

\textbf{Comentários}: If a mixture of fruit is used, the perry character must remain dominant. Regardless of experimental nature, the resulting beverage must be recognizable as a perry. The description of the perry is critical information for judges, and should be sufficient to allow them to understand the concept. If special ingredients are declared, they should be perceived (exception: potential allergens do not need to be perceivable, but must be declared).

\textbf{Instruções para Inscrição}: Entrants MUST specify the ingredients or processes that make the entry an experimental perry. Entrants MUST specify both carbonation and sweetness levels. Entrants MAY specify a base style, or provide a more detailed description of the concept.

\textbf{Varietais}: Any

\textbf{Estatísticas}: OG: 1.045 - 1.100 \\
\phantom{ } \hspace{16.5mm} FG: 0.995 - 1.020 \\
\phantom{ } \hspace{16.5mm} ABV: 5 - 12\%

\textbf{Exemplos Comerciais}: Æppeltreow Pear Wine, Sea Cider Ginger Perry, Snow Capped Cider JalaPEARño