\phantomsection
\subsection*{C2D. Fire Cider}
\addcontentsline{toc}{subsection}{C2D. Fire Cider}

\textit{Um estilo canadense de sidra (cidre de feu), feito com variedades clássicas de maçãs de mesa norte-americanas, fermentado a partir de suco fervido e concentrado. A fermentação pode ser intencionalmente interrompida ou parada enquanto ainda há uma quantidade substancial de açúcar residual. Nenhum aditivo é permitido; em particular, fontes de açúcar não podem ser usados para aumentar a densidade. Versões comerciais podem ser envelhecidas por até cinco anos antes de serem lançadas.}

\textbf{Impressões Gerais}: De dourado escuro a marrom, com uma impressão muito doce, caramelizada, semelhante a açúcar de bordo (\textit{maple sugar}). Versões bem envelhecidas frequentemente exibem caráter de frutas escuras ou lembram vinho Jerez (sherry).

\textbf{Aroma e Sabor}: Aroma doce e intenso de açúcares caramelizados, que podem lembrar xarope de bordo ou açúcar mascavo, com notas de caramelo, damascos secos, maçãs assadas ou butterscotch. Versões envelhecidas podem trazer notas de frutas escuras e caráter semelhante a Jerez. Um leve caráter defumado, se presente, não é considerado falha. Acidez e taninos geralmente são discretos no equilíbrio. Dulçor de alto a muito alto, mas sem ser enjoativo. Álcool até um nível moderado pode estar presente, mas precisa estar bem integrado. 

\textbf{Aparência}: De límpida a brilhante. Cor muito mais profunda que uma Common Cider ou Ice Cider, variando de dourado escuro a marrom.

\textbf{Sensação na Boca}: Corpo cheio, por vezes com uma alta viscosidade, densa. Alguns exemplos podem ter taninos moderados, mas nunca a ponto de parecerem adstringentes ou agressivos. Carbonatação geralmente de tranquila a moderada. O aquecimento alcoólico pode ser percebido de baixo a moderadamente baixo, às vezes menos evidente do que a força alcoólica sugeriria. Exemplares bem envelhecidos podem exibir uma maciez característica.

\textbf{Comentários}: A caramelização é desejável, mas sabores queimados ou torrados são uma falha. Difere da Ice Cider porque deve ter caráter de caramelização e a acidez geralmente é mais baixa no equilíbrio. 

\textbf{Instruções para Inscrição}: Participantes \textbf{DEVEM} especificar densidade inicial, densidade final ou açúcar residual, e teor alcoólico. Participantes \textbf{DEVEM} especificar o nível de carbonatação.

\textbf{Varietais}: Frutas de mesa clássicas norte-americanas como McIntosh, Cortland ou Spartan.

\textbf{Estatísticas}: OG: 1.130 - 1.180 \\
\phantom{ } \hspace{16.5mm} FG: 1.040 - 1.075 \\
\phantom{ } \hspace{16.5mm} ABV: 9 - 16\%

\textbf{Exemplos Comerciais}: Cideri Milton Cidre de Feu, Domain Labranche Fire Cider, Lacroix Feu Sacré, Petit et Fils Le Jaseux, Union Libre Fire Cider
