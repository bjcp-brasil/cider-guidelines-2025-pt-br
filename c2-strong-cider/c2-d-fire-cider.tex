\phantomsection
\subsection*{C2D. Fire Cider}
\addcontentsline{toc}{subsection}{C2D. Fire Cider}

\textit{A Canadian cider style (cidre de feu) using classic North American table fruit varietals, and fermented from boiled, concentrated juice. Fermentation may be intentionally arrested or stopped while a substantial amount of residual sugar is present. No additives are permitted; in particular, sweeteners may not be used to increase gravity. Commercial versions may be aged for up to five years prior to release.}

\textbf{Impressões Gerais}: A dark gold to brown cider with a very sweet, caramelized, maple sugar-like impression. Well-aged versions often exhibit a dark fruit or sherry-like character.

\textbf{Aroma e Sabor}: Deep, sweet aroma of caramelized sugars that can have a character like maple syrup or brown sugar, with hints of caramel, dried apricots, baked apples, or butterscotch. Aged versions may have elements of dark fruits and often exhibit a sherry-like character. A very light smoke-like character, if present, is not a fault. Acidity and tannins are typically restrained in the balance. High to very high sweetness, yet not cloying. Up to moderate alcohol may be present, but must be well-integrated.

\textbf{Aparência}: Clear to brilliant. Color is much deeper than a Common Cider or Ice Cider, ranging from deep gold to brown.

\textbf{Sensação na Boca}: Full body, sometimes with a thick, chewy viscosity. Some examples can have moderate tannin levels, but not to the point where they seem overly astringent or harsh. Carbonation typically still to moderate. Alcohol warmth may be perceived at a low to moderately-low level, sometimes less obvious than the strength would otherwise indicate. Well-aged examples can exhibit a characteristic smoothness.

\textbf{Comentários}: Caramelization is desirable, but scorched or burnt flavors are a fault. Differs from Ice Cider in that it should have a character from caramelization, and the acidity is generally lower in the balance.

\textbf{Instruções para Inscrição}: Entrants \textbf{MUST} specify starting gravity, final gravity or residual sugar, and alcohol level. Entrants \textbf{MUST} specify carbonation level.

\textbf{Varietais}: Classic North American table fruit such as McIntosh, Cortland, or Spartan

\textbf{Estatísticas}: OG: 1.130 - 1.180 \\
\phantom{ } \hspace{16.5mm} FG: 1.040 - 1.075 \\
\phantom{ } \hspace{16.5mm} ABV: 9 - 16\%

\textbf{Exemplos Comerciais}: Cideri Milton Cidre de Feu, Domain Labranche Fire Cider, Lacroix Feu Sacré, Petit et Fils Le Jaseux, Union Libre Fire Cider