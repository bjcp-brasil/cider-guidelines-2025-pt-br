\phantomsection
\subsection*{C2C. Ice Cider}
\addcontentsline{toc}{subsection}{C2C. Ice Cider}

\textit{Uma sidra fermentada a partir de suco concentrado, seja congelando a fruta antes da prensagem ou congelando o suco para a remoção da água. A fermentação para ou é interrompida antes de atingir a secura completa.}

\textbf{Aroma e Sabor}: Frutada, com profundidade e complexidade de sabor de maçã. Suave, rica, doce e semelhante a um vinho de sobremesa, mas com acidez equilibrando, como em um Sauternes ou outro vinho de sobremesa de alta qualidade. A acidez deve ser alta o suficiente para evitar que fique enjoativa. Tem caráter brilhante quando jovem. O envelhecimento pode trazer maior complexidade, com notas de frutas mais escuras e de açúcares, mas não deve parecer fortemente caramelizada. Acidez volátil perceptível, tipicamente percebida como acetona, é uma falha.

\textbf{Aparência}: Brilhante. A cor é mais profunda do que em uma sidra padrão, variando de dourado a âmbar. Exemplares envelhecidos podem apresentar tons mais escuros.

\textbf{Sensação na Boca}: Corpo cheio. Pode ter taninos (adstringentes ou amargos), mas geralmente de leves a moderados, embora níveis mais altos e equilibrados sejam aceitáveis. Pode apresentar aquecimento, mas não deve ser agressiva.

\textbf{Comentários}: O caráter difere da Applewine porque o processo da Ice Cider aumenta não só o açúcar (e, portando, o potencial alcoólico), mas também a acidez e todos os componentes de sabor da fruta, proporcionalmente. Difere da Fire Cider, pois não possui sabores profundamente caramelizados, mas apresenta acidez mais alta para equilibrar o dulçor. Nenhum aditivo é permitido neste estilo; em particular, fontes de açúcar não podem ser usados para aumentar a densidade. Esse estilo se originou em Quebec, nos anos 1990.

\textbf{Instruções para Inscrição}: Participantes \textbf{DEVEM} especificar a densidade inicial, a densidade final ou açúcar residual, e o teor alcoólico. Participantes \textbf{DEVEM} especificar o nível de carbonatação. 

\textbf{Varietais}: Normalmente frutas de mesa clássicas norte-americanas, como McIntosh ou Cortland.

\textbf{Estatísticas}: OG: 1.130 - 1.180 \\
\phantom{ } \hspace{16.5mm} FG: 1.050 - 1.085 \\
\phantom{ } \hspace{16.5mm} ABV: 7 - 13\%

\textbf{Exemplos Comerciais}: Champlain Orchards Honeycrisp Ice Cider, Cidrerie St-Nicolas Glace Du Verger Iced Orchard Cider, Domaine Pinnacle Cidre de Glace, Eden Heirloom Blend Ice Cider, Eve's Cider Essence, Les Vergers de la Colline Le Glacé, Windfall Orchard Ice Cider
