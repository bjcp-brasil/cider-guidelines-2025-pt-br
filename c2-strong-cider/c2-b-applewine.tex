\phantomsection
\subsection*{C2B. Applewine}
\addcontentsline{toc}{subsection}{C2B. Applewine}

\textit{A cider fermented with added neutral sugar that increases the starting gravity, and thus the resulting alcohol, to levels well above those typical for Common Cider. The amount of added sugar is greater than what could be used in other styles to compensate for low gravity. Uses no fruit other than apples, and uses only sugar to increase the starting gravity.}

\textbf{Impressões Gerais}: Typically presents like a dry white wine, with fruity and floral notes. Balanced, with low astringency and bitterness. Alcohol is typically noticeable.

\textbf{Aroma e Sabor}: Comparable to a Common Cider in apple character, fruity and floral. Cider character must be distinctive. Very dry to sweet, although often dry. Dry versions can be fairly neutral. Light to moderate yeast character acceptable. Alcohol usually noticeable but should not be harsh, hot, or burning. Acidity typically medium to high. Tannins low to none. The combination of acidity, alcohol, and dryness must not make the finish too hard and tight.

\textbf{Aparência}: Clear to brilliant. Straw to medium-gold. Cloudiness or hazes are inappropriate.

\textbf{Sensação na Boca}: Dry versions may seem lighter in body than other ciders, because higher alcohol levels are derived from sugar additions rather than juice. Carbonation may range from still to Champagne-like. Typically has a light alcohol warmth.

\textbf{Comentários}: Differs from a New England Cider by using flavorless adjuncts. Sugar is added for chaptalization, or increasing the gravity of the juice in order to create more alcohol; it is not intended to increase residual sweetness. Does not contain grapes or fruit other than apples. Not related to Apfelwein, which is a German word for cider. Fortified or distilled products should not be entered in this style. Some commercial examples may be labeled as applewine based on ABV levels and local laws; when seeking examples, pay attention to the profile, not the labeling.

\textbf{Instruções para Inscrição}: Entrants MUST specify both carbonation and sweetness levels.

\textbf{Varietais}: Same as Common Cider

\textbf{Estatísticas}: OG: 1.070 - 1.100 \\
\phantom{ } \hspace{16.5mm} FG: 0.995 - 1.020 \\
\phantom{ } \hspace{16.5mm} ABV: 9 - 12\%

\textbf{Exemplos Comerciais}: Æppeltreow Autumn Glory Apple Wine, McClure’s Sweet Apple Wine, 1911 Established Empire Dry Applewine
