\phantomsection
\subsection*{C2B. Applewine}
\addcontentsline{toc}{subsection}{C2B. Applewine}

\textit{Uma sidra fermentada com adição de açúcares neutros, que aumentam a densidade inicial e, consequentemente, o álcool resultante, para níveis bem acima dos típicos para uma Common Cider. A quantidade de açúcar adicionado é maior do que poderia ser usada em outros estilos para compensar a baixa densidade. Não é produzida com outras frutas além de maçãs e leva apenas açúcar para aumentar a densidade inicial.}

\textbf{Impressões Gerais}: Normalmente se apresenta como um vinho branco seco, com notas frutadas e florais. Equilibrada, com baixa adstringência e amargor. O álcool normalmente é perceptível.

\textbf{Aroma e Sabor}: Comparável a uma Common Cider no quesito caráter de maçã, frutado e floral. O caráter de sidra deve ser perceptível. Pode ir de muito seca a doce, embora seja frequentemente seca. Versões secas podem ser bastante neutras. Caráter de levedura de leve a moderado é aceitável. O álcool geralmente é perceptível, mas não deve ser agressivo, quente ou queimar. Acidez tipicamente de média a alta. Taninos de baixos a inexistentes. A combinação de acidez, álcool e secura não deve tornar o final difícil.

\textbf{Aparência}: De límpida a brilhante. Cor de palha a dourado médio. Turbidez é inapropriada.

\textbf{Sensação na Boca}: Versões secas podem aparentar corpo mais leve que outras sidras, pois os níveis mais altos de álcool vêm de adições de açúcar, não do suco. A carbonatação pode variar de tranquila a espumante. Normalmente apresenta um leve aquecimento alcoólico.

\textbf{Comentários}: Difere de uma New England Cider por usar adjuntos sem sabor. O açúcar é adicionado para chaptalização, ou seja, para aumentar a densidade do mosto e gerar mais álcool; não é destinado a aumentar o dulçor residual. Não contém uvas nem frutas além das maçãs. Não está relacionado ao \textit{Apfelwein}, que é a palavra alemã para sidra. Produtos fortificados ou destilados não devem ser inscritos neste estilo. Alguns exemplos comerciais podem ser rotulados como \textit{“applewine”} com base no teor alcoólico e nas leis locais; ao procurar exemplos, preste atenção ao perfil, não ao rótulo.

\textbf{Instruções para Inscrição}: Participantes \textbf{DEVEM} especificar tanto o nível de carbonatação quanto o de dulçor.

\textbf{Varietais}: Os mesmos da Common Cider.

\textbf{Estatísticas}: OG: 1.070 - 1.100 \\
\phantom{ } \hspace{16.5mm} FG: 0.995 - 1.020 \\
\phantom{ } \hspace{16.5mm} ABV: 9 - 12\%

\textbf{Exemplos Comerciais}: Æppeltreow Autumn Glory Apple Wine, McClure’s Sweet Apple Wine, 1911 Established Empire Dry Applewine
