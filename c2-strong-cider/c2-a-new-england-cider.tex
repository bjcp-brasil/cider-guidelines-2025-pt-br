\phantomsection
\subsection*{C2A. New England Cider}
\addcontentsline{toc}{subsection}{C2A. New England Cider}

\textit{This cider is made using traditional New England apples with relatively high acidity, and adjuncts to raise alcohol levels and contribute additional flavor notes. New England is a multi-state region in the northeast United States, east of New York state.}

\textbf{Impressões Gerais}: Substantial body and character. Typically is relatively dry, but can be somewhat sweet if in balance and not containing hot alcohol. Sometimes has a barrel character. Often has flavors from adjuncts, especially raisins.

\textbf{Aroma e Sabor}: A flavorful cider with robust apple character, strong but neutral alcohol, and derivative flavors from adjuncts and sugar additives. Traditionally dry, but sweetness can be present to balance stronger flavors. If sugary adjuncts have a flavor or aroma, those should be balanced with the apple flavor and not dominate. A raisin-like flavor is common. Any barrel or wood character should be restrained, not dominant. Tannins can add to the dryness of the finish. Acid levels are moderate to high, and should be in balance with other flavors. Alcohol should not be hot or harsh. There are many possible flavors present; the best examples show an integration and harmonization between components.

\textbf{Aparência}: Clear to brilliant. Yellow to amber color. Darker colors allowable with declared ingredients and barrel aging.

\textbf{Sensação na Boca}: Moderate to full body. Alcohol warmth typical, but should not have a hot character. Medium-low to moderate tannins, which can be higher if barrel-aged. Variable carbonation.

\textbf{Comentários}: Adjuncts may include any of white sugar, brown sugar, molasses, maple syrup, or small amounts of honey. Raisins are common. These adjuncts are intended to raise the OG well above what could be achieved by apples alone. Sometimes barrel-aged, which can add an oak character similar to a barrelaged wine. If the barrel previously held spirits, some of their flavor notes (e.g., whisky, rum) may be present, but must be subtle and balanced. New England Cider is a traditional style; do not interpret it to mean any cider from New England. It also has nothing to do with New England (Hazy) IPA.

\textbf{Instruções para Inscrição}: Entrants \textbf{MUST} specify if the cider was barrel-fermented or -aged. Entrants \textbf{MUST} specify both carbonation and sweetness levels.

\textbf{Varietais}: Traditional New England apples, such as Northern Spy, Roxbury Russet, Golden Russet, Baldwin

\textbf{Estatísticas}: OG: 1.060 - 1.100 \\
\phantom{ } \hspace{16.5mm} FG: 0.995 - 1.020 \\
\phantom{ } \hspace{16.5mm} ABV: 7 - 13\%

\textbf{Exemplos Comerciais}: Blackbird Cider Works New England Style, Doc’s New England Small Batch Cider, Dressler Estate Outpost, Gypsy Circus New England Pantomime, Tandem Ciders Scrumpy Little Woody