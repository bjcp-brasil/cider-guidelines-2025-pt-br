\phantomsection
\subsection*{C2A. New England Cider}
\addcontentsline{toc}{subsection}{C2A. New England Cider}

\textit{Esta sidra é produzida usando maçãs tradicionais da Nova Inglaterra (New England), com acidez relativamente alta, e adjuntos para aumentar os níveis de álcool e contribuir com notas adicionais de sabor. Nova Inglaterra é uma região multiestados no nordeste dos Estados Unidos, a leste do estado de Nova York.}

\textbf{Impressões Gerais}: Encorpada e com presença. Tipicamente é relativamente seca, mas pode ser um pouco doce se equilibrada e não apresentar aquecimento alcoólico. Por vezes pode apresentar caráter de barril. Frequentemente tem sabores vindos de adjuntos, especialmente uvas passas.

\textbf{Aroma e Sabor}: Uma sidra saborosa com caráter robusto de maçã, álcool forte, mas neutro, e sabores derivados de adjuntos e açúcares adicionados. Tradicionalmente seca, mas o dulçor pode estar presente para equilibrar sabores mais fortes. Se adjuntos açucarados tiverem sabor ou aroma, estes devem estar em equilíbrio com o sabor de maçã e não dominá-lo. Um sabor de uvas passas é comum. Qualquer caráter de barril ou madeira deve ser contido, não dominante. Taninos podem contribuir para a secura do final. Os níveis de acidez são de moderados a altos, devendo estar em equilíbrio com os outros sabores. O álcool não deve ser quente ou áspero. Há muitos sabores possíveis; os melhores exemplos mostram integração e harmonia entre os componentes.

\textbf{Aparência}: Límpida a brilhante. Cor de amarelo a âmbar. Tons mais escuros são aceitáveis se vierem de ingredientes declarados ou de barril.

\textbf{Sensação na Boca}: Corpo de moderado a cheio. Aquecimento alcoólico pode ser percebido, mas não deve ser agressivo. Taninos de médio-baixos a moderados, podendo ser mais altos se envelhecida em barril. Carbonatação variável.

\textbf{Comentários}: Adjuntos utilizados podem incluir qualquer um de açúcar branco, açúcar mascavo, melaço, xarope de bordo ou pequenas quantidades de mel. Uvas passas são comuns. Esses adjuntos têm como objetivo elevar a OG bem acima do que poderia ser alcançado apenas com maçãs. Às vezes envelhecida em barril, o que pode adicionar caráter de carvalho semelhante a um vinho envelhecido em barril. Se o barril anteriormente continha destilados, algumas de suas notas de sabor (por exemplo, uísque, rum) podem estar presentes, mas devem ser sutis e equilibradas. A New England Cider é um estilo tradicional; não deve ser interpretada como qualquer sidra apenas produzida na Nova Inglaterra. Também não tem nada a ver com New England (Hazy) IPA.

\textbf{Instruções para Inscrição}: Participantes \textbf{DEVEM} especificar se a sidra foi fermentada ou envelhecida em barril. Participantes \textbf{DEVEM} especificar níveis de carbonatação e dulçor.

\textbf{Varietais}: Maçãs tradicionais da Nova Inglaterra, como Northern Spy, Roxbury Russet, Golden Russet, Baldwin.

\textbf{Estatísticas}: OG: 1.060 - 1.100 \\
\phantom{ } \hspace{16.5mm} FG: 0.995 - 1.020 \\
\phantom{ } \hspace{16.5mm} ABV: 7 - 13\%

\textbf{Exemplos Comerciais}: Blackbird Cider Works New England Style, Doc’s New England Small Batch Cider, Dressler Estate Outpost, Gypsy Circus New England Pantomime, Tandem Ciders Scrumpy Little Woody
