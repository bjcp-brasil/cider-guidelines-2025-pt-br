\section*{C3. Specialty Cider}
\addcontentsline{toc}{section}{C3. Specialty Cider}
\textit{Specialty Cider englobam aquelas sidras que recebem adição de ingredientes para saborização ou que utilizam processos não descritos nas categorias C1 e C2. Essas sidras não precisam, necessariamente, declarar um estilo-base, como ocorre em alguns estilos de especialidade de cerveja, mas devem manter uma forma perceptível de sidra como base. O resultado deve ser reconhecível como uma sidra, em combinação harmoniosa do ingrediente especial, resultando em um produto final coerente e equilibrado. Declarar um estilo-base é permitido, mas, neste caso, é importante entender que os juízes também avaliarão a sidra em relação a esse estilo declarado. As amostras sem estilo-base declarado ainda assim devem apresentar caráter evidente de sidra.}.
\textit{No contexto desta categoria, o termo \textbf{fruta} é definido conforme o mesmo uso apresentado no preâmbulo da categoria 29 - Fruit Beer do Guia de Estilos de Cerveja 2021 do BJCP. Da mesma forma, o termo \textbf{especiaria} é definido segundo a categoria 30 - Spiced Beer, incluindo o uso de quaisquer especiarias, ervas ou vegetais.}
\textit{As mesmas características gerais e descrições de falhas se aplicam às Specialty Ciders, assim como às Traditional Ciders (categoria C1), com exceção dos ingredientes adicionais permitidos. Para mais detalhes sobre as características sensoriais aplicáveis a todos os estilos, consulte a Introdução aos Estilos de Sidra e Perada.}
