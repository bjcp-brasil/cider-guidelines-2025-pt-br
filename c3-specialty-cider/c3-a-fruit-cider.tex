\phantomsection
\subsection*{C3A. Fruit Cider}
\addcontentsline{toc}{subsection}{C3A. Fruit Cider}

\textit{Uma sidra com adição de outras frutas ou de seus sucos, que não maçã. Este é o estilo correto para inscrever uma bebida fermentada a partir da combinação de suco de maçã e de pera.}

\textbf{Impressão Geral}: Integração agradável entre a sidra e a fruta adicionada. Tanto o caráter de maçã quanto o da fruta adicionada devem ser perceptíveis, equilibrados e complementares. Se um estilo-base for declarado, deve apresentar evidências desse estilo. Se nenhum estilo-base for declarado, assume-se tratar de uma Common Cider.

\textbf{Aroma e Sabor}: O caráter de sidra deve estar presente e se integrar de forma harmoniosa com as outras frutas declaradas. O caráter frutado pode lembrar fruta fresca ou geleia, mas deve sempre apresentar qualidade fermentada. É considerado uma falha se a fruta adicionada dominar completamente a sidra, se a fruta parecer suco in natura não fermentado, ou se a sidra tiver sabor artificial. A oxidação da sidra-base ou da fruta é considerada falha, embora a sidra possa apresentar caráter fresco ou envelhecido.

\textbf{Aparência}: De límpida a brilhante, conforme apropriado ao estilo-base. A cor deve ser condizente com a fruta adicionada, sem apresentar características de oxidação, amarronzadas ou opacas (por exemplo, frutas vermelhas devem resultar em cor vermelha a púrpura, não alaranjada). As variedades de frutas podem apresentar uma gama de cores mais ampla do que a tradicionalmente associada à fruta declarada.

\textbf{Sensação na Boca}: Refletindo o estilo-base. Alguns ingredientes podem contribuir com acidez ou taninos adicionais.

\textbf{Comentários}: Normalmente produzida a partir de pelo menos 75\% de suco de maçã, mas esse valor não precisa ser declarado. O produto final deve preservar o caráter de sidra proveniente das maçãs. A descrição da sidra é uma informação crítica para os juízes e deve ser suficiente para que compreendam o conceito. Se ingredientes especiais forem declarados, eles devem ser perceptíveis (exceção: potenciais alérgenos não precisam ser perceptíveis, mas devem ser declarados).

\textbf{Instruções para Inscrição}: Participantes \textbf{DEVEM} especificar tanto os níveis de carbonatação quanto de dulçor. Participantes \textbf{DEVEM} especificar todas as frutas ou sucos de frutas adicionados. Participantes \textbf{PODEM} especificar um estilo-base de sidra. Participantes \textbf{PODEM} especificar a cor da fruta adicionada.

\textbf{Varietais}: Qualquer, dependendo da sidra-base.

\textbf{Estatísticas}: OG: 1.045 - 1.070 \\
\phantom{ } \hspace{16.5mm} FG: 0.995 - 1.010 \\
\phantom{ } \hspace{16.5mm} ABV: 5 - 9\%

\textbf{Exemplos Comerciais}: Apple Valley Black Currant, Bauman's Cider Loganberry, Tandem Ciders Strawberry Jam, Tieton Cranberry, Uncle John's Apple Cherry Hard Cider, Vander Mill Bluish Gold
