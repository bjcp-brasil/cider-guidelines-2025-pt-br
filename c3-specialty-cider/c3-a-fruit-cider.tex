\phantomsection
\subsection*{C3A. Fruit Cider}
\addcontentsline{toc}{subsection}{C3A. Fruit Cider}

\textit{A cider with additional non-apple fruit or fruit juices added. This is the correct style to enter a beverage fermented from a combination of apple and pear juice.}

\textbf{Impressões Gerais}: A pleasant integration of cider and added fruit. The apple character and the added fruit must be noticeable, balanced, and complementary. If a base style is declared, should show some evidence of that style. If no base style is declared, assume to be a Common Cider.

\textbf{Aroma e Sabor}: The cider character must be present and must meld well with the other declared fruits. The fruit character can seem like fresh fruit or somewhat jam-like, but should always have a fermented quality. It is a fault if the added fruit completely dominates the cider, the fruit seems like raw unfermented juice, or the cider otherwise tastes artificial. Oxidation of the base cider or of the fruit is a fault, but the cider can have a fresh or aged character.

\textbf{Aparência}: Clear to brilliant, as appropriate for the base style. Color appropriate to added fruit, but should not show brownish or dull oxidation characteristics (for example, red berries should give red-to-purple color, not orange). Fruit varieties can come in a range of colors often broader than those traditionally associated with the declared fruit.

\textbf{Sensação na Boca}: Reflecting base style. Some ingredients may contribute additional acidity or tannins.

\textbf{Comentários}: Typically made from at least 75\% apple juice, but this value does not need to be declared. The final product must retain a cider character from the apples. The description of the cider is critical information for judges, and should be sufficient for them to understand the concept. If special ingredients are declared, they should be perceived (exception: potential allergens do not need to be perceivable, but must be declared).

\textbf{Instruções para Inscrição}: Entrants MUST specify both carbonation and sweetness levels. Entrants MUST specify all fruit or fruit juice added. Entrants MAY specify a base cider style. Entrants MAY specify the color of added fruit.

\textbf{Varietais}: Any, depending on base cider

\textbf{Estatísticas}: OG: 1.045 - 1.070 \\
\phantom{ } \hspace{16.5mm} FG: 0.995 - 1.010 \\
\phantom{ } \hspace{16.5mm} ABV: 5 - 9\%

\textbf{Exemplos Comerciais}: Apple Valley Black Currant, Bauman's Cider Loganberry, Tandem Ciders Strawberry Jam, Tieton Cranberry, Uncle John's Apple Cherry Hard Cider, Vander Mill Bluish Gold