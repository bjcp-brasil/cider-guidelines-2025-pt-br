\phantomsection
\subsection*{C3C. Experimental Cider}
\addcontentsline{toc}{subsection}{C3C. Experimental Cider}

\textit{Esta é uma categoria aberta e abrangente para sidras com outros ingredientes ou processos que não se encaixem em nenhum dos estilos de sidra das categorias C1 a C3. Também pode ser usada para qualquer outro tipo de sidra histórica ou regional tradicional ainda não descrita. Se a sidra se enquadra em um estilo já definido, então não é considerada uma Experimental Cider.}

\textbf{Aroma e Sabor}: The cider character must always be present, and must fit with added ingredients or process effects. If a spirit barrel was used, the character of the spirit (rum, whiskey, etc.) may range from subtle (barely recognizable) to balanced and complementary (short of dominating and overwhelming the cider character). Overall balance and drinkability are the critical success factors for this style. The resulting cider should contain recognizable experimental components, and be pleasant to drink.
\textbf{Aroma e Sabor}: O caráter de sidra deve estar sempre presente e deve harmonizar com os ingredientes adicionados ou os efeitos do processo. Se um barril previamente usado para destilados for utilizado, o caráter do destilado (rum, uísque, etc.) pode variar de sutil (quase imperceptível) até equilibrado e complementar (sem dominar ou sobrepujar o caráter da sidra). O equilíbrio geral e a facilidade em beber são fatores críticos de sucesso neste estilo. A sidra resultante deve conter elementos experimentais reconhecíveis e ser agradável de beber.

\textbf{Aparência}: De límpida a brilhante, conforme apropriado ao estilo-base. A cor deve ser a de uma sidra padrão, a menos que outros ingredientes ou processos contribuam intencionalmente para a cor.

\textbf{Sensação na Boca}: Reflete o estilo-base, mas também pode apresentar características tânicas, adstringentes, amargas, corpo mais pesado ou outros atributos determinados pelos ingredientes ou processos declarados.

\textbf{Comentários}: Alguns exemplos que se enquadram nesta categoria incluem:
\begin{itemize}[leftmargin=3mm]
\item Sidra com adição de mel (a menos que seja usada no estilo \textit{New England Cider}, ou se o mel for dominante no equilíbrio, caso em que deve ser inscrita como M2A \textit{Cyser} segundo as Diretrizes de Hidromel)
\item Sidra com outros adoçantes
\item Sidras com especiarias e outras frutas (não maçãs)
\item Híbridos de sidra e cerveja (\textit{graff/graf, snakebite})
\item Sidra com caráter de madeira ou barril como parte significativa do perfil de sabor
\item Sidra que, embora atenda às definições existentes das diretrizes, esteja visivelmente fora dos parâmetros listados (ex.: teor alcoólico, dulçor, carbonatação)
\item Estilos regionais, tradicionais ou históricos não contemplados nas diretrizes.
\end{itemize}
Independentemente da natureza experimental, a bebida resultante deve ser reconhecível como sidra. A descrição da sidra é uma informação crítica para os juízes e deve ser suficiente para que compreendam o conceito. Se ingredientes especiais forem declarados, eles devem ser perceptíveis (exceção: potenciais alergênicos não precisam ser perceptíveis, mas devem ser declarados).  
A Sidra Experimental pode exceder os intervalos típicos das Estatísticas para os estilos-base declarados, especialmente quando baseada em estilos concentrados (C2C ou C2D).

\textbf{Instruções para Inscrição}: Os participantes \textbf{DEVEM} especificar os ingredientes ou processos que tornam a amostra uma sidra experimental. Os participantes \textbf{DEVEM} especificar os níveis de carbonatação e dulçor. Os participantes \textbf{PODEM} especificar um estilo-base ou fornecer uma descrição mais detalhada do conceito.

\textbf{Varietais}: Qualquer uma, dependendo da sidra base.

\textbf{Estatísticas}: OG: 1.045 - 1.100 \\
\phantom{ } \hspace{16.5mm} FG: 0.995 - 1.020 \\
\phantom{ } \hspace{16.5mm} ABV: 5 - 12\%

\textbf{Exemplos Comerciais}: Cidergeist Beezy, Domaine Dupoint Cidre Reserve, Finnriver Fire Barrel, Snowdrift Cornice, Tandem Ciders Bee's Dream, Uncle John's Blossom Blend, Uncle John's Sidra de Tepache