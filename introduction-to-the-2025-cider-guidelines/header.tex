\section*{Introdução aos Estilos de Sidra e Perada (Categorias C1-C4)}
\addcontentsline{toc}{section}{Introdução aos Estilos de Sidra e Perada (Categorias C1-C4)}

\textit{\textbf{Sidra}, do inglês \textbf{Cider}, é a bebida fermentada a partir do suco de maçãs moídas. \textbf{Perada}, do inglês \textbf{Perry}, é uma bebida semelhante, feita a partir de peras. Nos Estados Unidos, faz-se uma distinção entre “hard cider” (fermentada, com álcool) e “sweet cider” (não fermentada, sem álcool). Em outras partes do mundo, sidra se refere ao produto fermentado – usamos esta definição neste guia.}

\textit{Há quatro categorias para sidra e perada neste guia: Traditional Cider (Categoria C1), Strong Cider (Categoria C2), Specialty Cider (Categoria C3) e Perry (Categoria C4). Veja a introdução de cada categoria para descrições mais detalhadas. Assim como na cerveja, não há exigência de que competições julguem essas categorias separadamente – estilos individuais podem ser agrupados para fins de julgamento e premiação. Não tente inferir qualquer significado mais profundo dos nomes ou agrupamentos, pois não foram pensados com esta intenção.}