\phantomsection
\subsection*{Aroma e Sabor}
\addcontentsline{toc}{subsection}{Aroma e Sabor}

\begin{itemize}
\item Sidras e peradas não necessariamente apresentam aromas ou sabores fortemente frutados — da mesma forma que o vinho não tem gosto de suco de uva ou que a cerveja não tem cheiro de mosto. Estilos mais secos de sidra podem desenvolver um caráter mais complexo, mas menos frutado. Sidras e peradas não devem ter gosto de um coquetel de suco de fruta, água com gás e álcool – elas devem ter gosto de bebida fermentada.
\item Os produtores de vinhos classificam os cheiros como aroma, aqueles derivados dos ingredientes, ou buquê, aqueles derivados do processo de fermentação e envelhecimento. Juízes de sidra podem se beneficiar pensando de forma semelhante, entendendo como o processo de produção transforma os ingredientes no produto final.
\item Um perfil de fermentação limpo é desejável na maioria dos estilos, mas isso não significa necessariamente a ausência de \textbf{caráter de levedura}. A levedura pode adicionar notas esterificadas ou pode ter um frescor levemente sulfuroso; estes não são defeitos. O envelhecimento em contato com a levedura pode contribuir com leves notas tostadas, de nozes ou que remetem a pão.
\item Alguns estilos de sidra apresentam características claramente não frutadas, como as notas de presunto defumado de uma English Cider seca. Alguns estilos regionais podem apresentar um caráter rústico.
\item \textbf{Dulçor} (açúcar residual, ou \textbf{RS}, de “residual sugar”), varia de absolutamente seco (sem RS) até níveis comparáveis a vinhos de sobremesa (RS igual ou maior a 10\%). Medidas aproximadas de RS e densidade final (FG) para os níveis de dulçor são:
  \begin{itemize}
  \item[o] \textbf{Seco}: RS abaixo de 0,4\% RS, FG menor que 1.002. Nenhuma percepção de dulçor, mas a percepção não precisa ser extremamente seca.
  \item[o] \textbf{Meio-seco}: 0,4–0,9\% RS, FG 1.002–1.004. Há um leve toque de dulçor, mas a percepção ainda é predominantemente seca. Também conhecido como semi-seco ou “off-dry”.
  \item[o] \textbf{Médio}: 0,9–2,0\% RS, FG 1.004–1.009. O dulçor agora é um componente notável do equilíbrio geral.
  \item[o] \textbf{Meio-doce}: 2,0–4,0\% RS, FG 1.009–1.019. A percepção é doce, mas ainda refrescante. Também conhecido como semi-doce.
  \item[o] \textbf{Doce}: RS acima de 4,0\%, FG acima de 1.019. Similar a um vinho de sobremesa. Não deve parecer como um xarope, nem ser enjoativo.
  \end{itemize}
Esses números servem para ajudar a decidir como inscrever a bebida e para normalizar diferenças de percepção regionais, não devendo ser usados como fator de desclassificação pelos juízes. Quando próximo dos limites entre níveis de dulçor, a inscrição deve ser baseada na impressão geral e em como a bebida corresponde às descrições desses níveis.
Esteja ciente de que outros fatores (acidez, taninos, álcool, secura, outros ingredientes etc.) afetam a percepção de dulçor. Não se baseie apenas nos níveis de RS.
Ao julgar, organize as amostras em ordem crescente de dulçor. Entenda que o dulçor pode mascarar defeitos — esteja mais alerta a isso em sidras mais doces. Da mesma forma, não penalize excessivamente sidras secas por falhas menores que podem estar mais evidentes pela falta de dulçor.
Em exemplares mais doces, componentes não frutados do sabor — particularmente acidez e taninos — devem complementar o dulçor, ou parecerão enjoativos (como xarope, excessivamente doces) ou sem estrutura (dulçor desequilibrado pela acidez).
\item \textbf{Acidez} é um elemento essencial do equilíbrio, conferindo uma impressão clara, viva, brilhante, suculenta e refrescante sem ser excessivamente agressiva. A acidez (do ácido málico e, em alguns casos, do ácido láctico ou outros) não deve ser confundida com acetificação (do acetato de etila ou ácido acético — vinagre). O aroma acre e o sabor picante da acidez volátil (acetificação) são considerados defeitos na maioria dos estilos.
\item \textbf{Tanino} fornece adstringência, corpo e, às vezes, amargor, o que contribui para o equilíbrio e estrutura da bebida e a facilidade de bebê-la. Amargor excessivo de taninos é um defeito, seja vindo do processo ou dos ingredientes utilizados. Frutas de mesa normalmente têm baixos teores de taninos.
\item As sidras podem passar por uma \textbf{fermentação maloláctica} (MLF), que reduz a acidez ao converter o ácido málico, de sabor mais pungente, em ácido láctico, mais suave e arredondado. O resultado não deve ser uma bebida sem estrutura ou excessivamente suave – a sidra deve permanecer refrescante. As peradas não devem passar por MLF porque o processo pode resultar em acetificação. A MLF pode produzir sabores mais limpos, mas o processo em maçãs ricas em taninos geralmente produz etilfenóis, com sabores descritos como picantes, defumados, carne defumada, fenólicos, de celeiro, funky, de couro ou de cavalo. Não espere a presença de todos esses descritores simultaneamente. Níveis moderados e equilibrados deles são opcionais, mas desejáveis em alguns estilos regionais. A MLF é frequentemente confundida com a presença de Brettanomyces (Brett), já que compartilham muitos descritores, mas a contaminação por Brett é um defeito sério. Um caráter dominante de celeiro, funky, proveniente de Brett, é indesejável. Juízes devem estar atentos à possível falha que remete a rato (“mousy”, THP, tetraidropiridina), mais provável de acontecer em uma sidra de pH elevado que tenha passado por MLF (para juízes incapazes de detectar essa falha, pode ser necessário um enxágue oral alcalino para confirmar).
\end{itemize}
