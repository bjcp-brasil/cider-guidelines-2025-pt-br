\phantomsection
\subsection*{Sensação na boca}
\addcontentsline{toc}{subsection}{Sensação na boca}
\begin{itemize}
\item Em geral, sidras e peradas têm corpo e preenchimento de boca semelhantes a um vinho leve. A maioria dos estilos de sidra tem muito menos corpo do que a maioria das cervejas. Algumas peradas terão corpos mais cheios devido ao sorbitol (um álcool de açúcar não fermentável), que também pode adicionar uma percepção de dulçor.
\item Sidras altamente espumantes podem lembrar um Champagne. Sidras tranquilas podem parecer “sem graça” para iniciantes, já que a carbonatação dá vida à apresentação. Sidras devidamente declaradas como tranquilas não devem ser penalizadas pela falta de carbonatação.
\item Os taninos podem afetar a sensação na boca ao adicionar corpo, amargor ou ao aumentar a percepção de secura no final da bebida, ao engolí-la. Estilos tânicos podem ter uma sensação na boca agradavelmente adstringente, semelhante a um vinho tinto. Descritores de vinho como áspero, que resseca ou que prende (do inglês, \textit{grippy}) podem se aplicar. Uma impressão de madeira, couro, folhas secas ou cascas de maçã também pode estar presente, acompanhada dos correspondentes sabores.
\end{itemize}
