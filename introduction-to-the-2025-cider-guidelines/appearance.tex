\phantomsection
\subsection*{Aparência}
\addcontentsline{toc}{subsection}{Aparência}

\begin{itemize}
\item A limpidez pode variar de boa a brilhante. A falta de brilho cristalino não é um defeito, mas partículas visíveis são indesejáveis. Em alguns estilos, uma aparência rústica, com falta de brilho, é comum. As peradas são notoriamente difíceis de clarificar; como resultado, uma leve turbidez não é considerada um defeito. Entretanto, um reflexo meio oleoso na superfície, tanto na sidra quanto na perada, geralmente indica os estágios iniciais de contaminação láctica e é um defeito evidente.
\item A carbonatação pode variar de nenhuma a algo semelhante a refrigerante. Pouca ou nenhuma carbonatação é chamada de \textbf{tranquila} (do inglês, still), mas pode apresentar uma leve cócega na língua – não precisa ser totalmente sem gás. Carbonatação moderada é chamada de \textbf{frisante} (do inglês, pétillant). Altamente carbonatada é chamada de \textbf{espumante} (do inglês, sparkling). Nos níveis mais altos de carbonatação, pode haver formação de espuma, mas de baixa retenção. Entretanto, jorros, espuma excessiva e difícil de controlar são defeitos.
\item Uma sidra ou perada sem ingredientes adicionais é frequentemente de cor clara, tipicamente de palha a dourada. Note que algumas maçãs de polpa vermelha, como a Redfield, dão um tom rosado que não deve ser interpretado erroneamente como proveniente de outra fruta; em caso de dúvida, verifique as variedades de maçãs declaradas. Tons opacos ou amarronzados podem ser um indicativo de oxidação, embora tonalidades mais escuras possam vir do uso de maçãs de baixa acidez, do processo de keeving, do envelhecimento ou fermentação em madeira, do uso de processos de concentração, entre outros. Não assuma automaticamente a oxidação apenas pela cor. Obviamente, exemplares que contêm ingredientes adicionados geralmente refletem a cor dessas adições.
\end{itemize}
