\phantomsection
\subsection*{Ingredientes}
\addcontentsline{toc}{subsection}{Ingredientes}

\begin{itemize}
\item As variedades citadas das frutas servem apenas como exemplos comuns, não como uma exigência para os produtores. As frutas são divididas em (1) variedades de mesa, de consumo, culinárias ou de sobremesa, ou (2) maçãs para sidra ou peras para perada (variedades tradicionais ou especiais que não são boas para consumo direto). Frutas deste segundo grupo podem apresentar uma ampla gama de características não frutadas, que não devem ser confundidas com caráter de fermentação.
\item As maçãs usadas na produção de sidra são comumente classificadas por sua acidez e nível de taninos: \textbf{Doce} (do inglês, \textit{sweet}; baixa acidez, baixo tanino), \textbf{Ácida} (do inglês, \textit{sharp}; alta acidez, baixo tanino), \textbf{Doce-amarga} (do inglês, \textit{bittersweet}; baixa acidez, alto tanino) ou \textbf{Ácida-amarga} (do inglês, \textit{bittersharp}; alta acidez, alto tanino).
\item A levedura pode ser \textit{natural}, que ocorre na própria fruta ou está presente nos equipamentos de moagem e prensagem, ou \textit{selecionada}, de cultura adicionada pelo produtor de sidra e perada. A fermentação malolátcica (MLF) é permitida usando qualquer um desses métodos.
\item Em geral, adjuntos e aditivos são proibidos, exceto se especificamente permitidos em estilos particulares, ou apenas para corrigir baixos níveis iniciais de açúcares da maçã, a fim de produzir um produto estável. Quando usados, devem ser declarados. Açúcares neutros são permitidos como adjunto na maioria dos estilos, seja para ajustar a densidade inicial ou final. Adições de mel geralmente resultam em uma C3C Experimental Cider, uma C4D Experimental Perry ou um M2A Cyser (ver Guia de Estilos de Hidromel). Revise as descrições individuais dos estilos para quaisquer adjuntos permitidos ou proibidos.
\item Aditivos de processo e enzimas geralmente são permitidos, desde que não sejam perceptíveis no produto final. Enzimas podem ser usadas para clarificar o suco das frutas antes da fermentação. Ácido málico pode ser adicionado a sucos de baixa acidez para elevá-la a um nível seguro (tipicamente pH igual ou menor que 3,8), evitando contaminação bacteriana e aromas e sabores indesejados.
\item Sulfitos podem ser usados para controle microbiológico, mas o nível máximo aceito como seguro (200 mg/L) deve ser rigorosamente observado. Qualquer excesso de sulfito, detectável como cheiro de fósforo queimado no produto final, é uma falha grave.
\item Sorbato pode ser adicionado no envase para estabilizar a sidra. Entretanto, qualquer aroma ou sabor residual do uso excessivo de sorbato, como notas de gerânio, é problemático.
\item Dulçor residual pode ser obtido interrompendo a fermentação ou adicionando fontes de açúcar ou suco fresco. Produtos adoçados após a fermentação devem ser estáveis. Turbidez, \textit{gushing} ou espuma resultantes de fermentação pós-envase são considerados falhas graves.
\item A fermentação em barris de carvalho é um método tradicional para muitos estilos de sidra. Tais barris são reutilizados, então um caráter forte e fresco de carvalho não é esperado no produto final. Madeiras relativamente neutras podem ser usadas para fermentar ou envelhecer qualquer estilo de sidra ou perada. No entanto, isso significa que qualquer caráter de madeira no produto final deve ser apenas sutil, de fundo. O uso de madeira não implica automaticamente em um estilo de especialidade ou experimental, mas sim, a intensidade desse caráter.
\item Exemplares com caráter substancial de madeira ou barril devem ser inscritos como C3C Experimental Cider ou C4D Experimental Perry, a menos que o estilo especificamente permita. Ao usar madeira dessa forma, declare a espécie de madeira e o processo utilizado (ex.: barril, lascas, aduelas, tiras, espirais).
\end{itemize}
