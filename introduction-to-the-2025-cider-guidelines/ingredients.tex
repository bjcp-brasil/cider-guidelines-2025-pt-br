\phantomsection
\subsection*{Ingredients}
\addcontentsline{toc}{subsection}{Ingredients}

\begin{itemize}
\item Cited fruit varieties are meant to illustrate commonlyused examples, not dictate requirements for producers. Fruit are divided into (1) table, eating, culinary, or dessert varieties, or (2) cider apples or perry pears (heirloom or specialty varieties that do not make for good eating). Fruit in this second group may exhibit a wide range of non-fruity traits, which should not be confused with fermentation character.
\item Apples used in cider-making are commonly classified by acidity and tannin: \textbf{Sweet} (low acidity, low tannin), \textbf{Sharp} (high acidity, low tannin), \textbf{Bittersweet} (low acidity, high tannin), or \textbf{Bittersharp} (high acidity, high tannin).
\item Yeast may be either \textit{natural} (occurring on the fruit itself or present in the milling and pressing equipment) or \textit{cultured} (added by the cidermaker). MLF is allowed using either of these methods.
\item In general, adjuncts and additives are prohibited except where specifically allowed in particular styles, or if only to correct low starting levels of apple sugars in order to produce a stable product. When used, they must be declared. Neutral sugar is allowed as an adjunct in most styles either to adjust starting or finishing gravity. Honey additions generally result in either a C3C Experimental Cider, a C4D Experimental Perry, or an M2A Cyser (see Mead Style Guidelines). Review individual style descriptions for any allowed or prohibited adjuncts.
\item Common processing aids and enzymes are generally allowed as long as they are not perceivable in the finished product. Enzymes may be used to clarify juice before fermentation. Malic acid may be added to lowacid juice to raise acidity to a level safe for avoiding bacterial contamination and off-flavors (typically pH 3.8 or below).
\item Sulfites may be used for microbiological control, but the maximum accepted safe level (200 mg/l) must be strictly observed. Any excess sulfite detectable as burnt match in the finished product is a serious fault.
\item Sorbate may be added at bottling to stabilize the cider. However, any residual aroma or flavor from overuse of sorbate (e.g., a geranium note) is problematic.
\item Residual sweetness may be obtained by arresting fermentation, or by adding sweeteners or fresh juice. Back-sweetened products must be stable. Turbidity, gushing, or foaming resulting from post-packaging fermentation are considered serious faults.
\item Barrel fermentation in oak is a traditional method for many cider styles, but those barrels are reused so a strong, fresh oak character is not expected in the final product. Relatively neutral wood may be used to ferment or age any style. However, this means that any wood character in the finished cider must be at no more than a background level. The use of wood does not automatically imply a specialty or experimental style; however, the intensity of the wood character does.
\item Examples with a substantial wood or barrel character should be entered in either C3C Experimental Cider or C4D Experimental Perry, unless the style specifically allows it. When using wood in this manner, declare the species of wood, and the process used (e.g., barrel, chips, staves, strips, spirals).
\end{itemize}